\documentclass[conference]{IEEEtran} 

%2019 hl ref: https://en.wikibooks.org/wiki/LaTeX/Mathematics#Matrices_and_arrays
\ifCLASSINFOpdf 
\else 
\fi  
\hyphenation{op-tical net-works semi-conduc-tor}

%\usepackage{amsmath} 
%HL 2018-1-21: not sure if IEEE accepts this

\usepackage{graphicx}
\graphicspath{images/} %HL 2018-1-21 add this image folder 
\usepackage{float} % HL to force latex figure location

\begin{document}
 
% paper title 

\title{{\small Engineering Application Note 105bFD100 Deep Reinforcement Learning 6 DoF Robot} \\
Deep Reinfocement Learning \\For Robotics Manipulation (II) \\
(Under Construction) 
}
 
\author{
\IEEEauthorblockN{Harry Li $^{\ddagger}$, Ph.D.}
\IEEEauthorblockA{ 
Computer Engineering Department, San Jose State University\\
San Jose, CA 95192, USA} \\
Email$^{\dagger}$: harry.li@ctione.com\\
}

% make the title area
\maketitle
\begin{abstract}
This note describes the unity implementation of the 
deep reinfocement learning for robotics system control. 
The objectives of this notes is to provide an implementation
reference for state space, action space and reward
functions. In addition, this note builds the 
implementation reference with the connection to the 
published work by Google deepmind team.
\end{abstract}

% no keywords 
\IEEEpeerreviewmaketitle

\section{Introduction}

This note is prepared based on my discussion notes with 
CTI One intern team and SJSU graduate students. The 
core material to form this discussion is from the 
collection of a set of 4 document:  

1. Harry Li's meeting notes, unity-ai-2021-3-12.odt 
(harry@workstation:/media/harry/ easystore/backup-2020-2-15/
SJSU/CMPE258/258-1-lec/lec7-unity/unity); 

2. Harry Li's deep reinforcement learning white paper
(Part I on Deep Reinforcement Learning); 

3. Robot Arm training github repository, 
https://github.com/rkandas/RobotArmMLAgentUnity; 

4. Published paper by Google Deepmind team. 

5. Tutorial on 6 DoF with ML model. 
\begin{verbatim}
https://medium.com/xrpractices/how-to-train-
your-robot-arm-fbf5dcd807e1
\end{verbatim}

Our objective of this review is to simulate 
6 Degree of Freedom (DoF) model which has 
4 Bends, 1 Rotate, and 1 end effector (Gripper)
as shown in the figure to realize 

1. reaching a random target pose; 

2. pick \& place an object. 

Note, the ML training scope is to teach the robot arm reach 
the target component from any state. However, the arm should 
not go below the ground and the target object is always 
placed above the ground and within the reach of the robotic 
arm. We will discuss the implementation in the following area

1. the configuration process to carry out the 
experiment in Unity environment; 

2. the policies design;  

The goal beyond the review of the research paper
is to carry out 
the real-world experiments in our lab environment on 
FD100 robot to demonstrate our designed polices and 
smooth transation to a real world environment for fast 
readily deployable results. 
\DeclareGraphicsExtensions{.eps} %.eps
%\begin{figure}[Htb] 
\begin{figure}[H] 
\centering
\includegraphics[scale = 0.35]{figure1}  
\caption{An end effector 
(a set of three in this case) from CTI's FD100 robot below, whose
movement forms a trajectory in 3D space". }
\label{fig1} 
\end{figure} 
 
The bigger view of the FD100 robot with an end effector 
(a set of three in this case) in the view. 

\DeclareGraphicsExtensions{.eps} %.eps
%\begin{figure}[Htb] 
\begin{figure}[H] 
\centering
\includegraphics[scale = 0.35]{figure2} %name of the eps figure file 
\caption{The bigger view of the FD100 robot with an end effector 
(a set of three in this case) in the view.}
\label{fig2} 
\end{figure}

\section{Experiment Implementation}

The material presented here is based on the github
material mentioned in the above section. 

\subsection{Implementation of State, Action and Reward Functions}

Step 1. Create state table; 
For the 6 DoF robot, we have 

{\footnotesize 
Axis 1: the bottom-most axis which rotate 0 to 360 degrees 
[Rotate state]; 

Axis 2: the first bend axis from -90 to 90 degrees [Bend 
state 1]; 

Axis 3: second bend axis from -120 to 120 degrees [Bend
state 2];

Axis 4: third bend axis from -90 to 90 degrees [Bend
state 3];

Axis 5: fourth bend axis from -90 to 90 degrees [Bend
state 4]; 

Axis 6: the gripper, Open or Close [Gripper state] 
} 
\DeclareGraphicsExtensions{.eps} %.eps
%\begin{figure}[Htb] 
\begin{figure}[H] 
\centering
\includegraphics[scale = 0.35]{figure1-6dof-robot} %name of the eps figure file 
\caption{The robot joints diagram.}
\label{figure1-6dof-robot} 
\end{figure}
The figure is from the Turin Robot User Guide, pp. 23. 

Step 2. Create action table. You can 
find a discrete set of possible actions to take and build a 
table (a list) of them one by one. For example for driving 
a car game, steer to left, to rigth, and keep straight are
a set of 3 actions. Now for the continued action, for 
example, action left can be characterized as fuzzy membership
function which gives a range [0,1] to describe how much 
left. So in this case, we will use Deep Deterministic 
Policy Gradient (DDPG) technique for this type of
off-policy methods. The latest development in this area
is the Soft Actor Critic (SAC) technique. 

Step 3. Create reward (e.g. cost) function table to link each 
state to its corrensponding action. 
The reward for each action should be designed. 
The reward could be positive or negative (Penalty) in 
numerical value by intuition. Generally, normalized 
in [-1, 1] range. Here is the 3-point guidelines. 

{\bf Definition 1. Reward Guideline} 

{\it 
(1) When the robot exhibits the desired behaviour, 
the reward function $r_i > 0$, 

(2) When the robot deviates from the desired behaviour, 
the reward function $r_i < 0$,  

(3) When the robot gets into 
non-recoverable state like its arm causes harm, such as
hitting the ground, the reward function $r_i = -1$, 
e.g., hefty penalty, 

(4) When the robot reaches the target, 
the reward function $r_i = 1$, 
e.g., hefty reward.   
}

Step 4. Create Q-table to link state, action, and reward functions 
together as illustrated in the example below. 

Example: Suppose driving a video game car, one 
can drive the car to left, right, or keep straight. So 
the following Q-table can be generated. 
\DeclareGraphicsExtensions{.eps} %.eps
%\begin{figure}[Htb] 
\begin{figure}[H] 
\centering
\includegraphics[scale = 0.35]{figure1-qtable}  
\caption{Create Q-table to link state, action, 
and reward functions.}
\label{fig1-qtable} 
\end{figure}
In this figure the actions and states are for 
driving a video game car. We will need to look into the 
similar table for 6 DoF robot. 
From this reference
https://blogs.unity3d.com/2020/11/19/
robotics-simulation-in-unity-is-as-easy-
as-1-2-3\_ga=2.25222392.499619345.1615574223-1264198522.
1609136839, you can build a simulator to visualize 
your 6 DoF robot motion. 

\subsection{Resources for 6 DoF ML and Simulation}
The resouces for 6 DoF ML and simulation: 

1. Pick-and-place tutorial on GitHub.
\begin{verbatim}
https://github.com/Unity-Technologies/
Unity-Robotics-Hub/blob/
main/tutorials/pick_and_place/README.md
#part-3-naive-pick--place. 
\end{verbatim}

2. Unity Robotics Hub on GitHub. 
https://github.com/Unity-Technologies/Unity-Robotics-Hub. 

3. Train computer vision systems using Unity, see 
the computer vision blog series. 
\begin{verbatim}
https://blogs.unity3d.com/2020/05/01/
synthetic-data-simulating-myriad-
possibilities-to-train-
robust-machine-learning-models/ .
\end{verbatim}

4. Unity official robotics page. 
\begin{verbatim}
https://unity.com/solutions/automotive-
transportation-
manufacturing/robotics
_ga=2.226719515.725964444.1615792426
-1264198522.1609136839 
\end{verbatim}
You can also contact unity team directly with questions, 
feedback, or suggestions, at unity-robotics@unity3d.com. 

\section{Experiment Design}
Unity’s ML-agents training framework is built based on 
the Markov Decision Process (MDP) which for an ML-agent, 
e.g., the 6 DoF robot works in the following manner for
each time step: 

Step 1. The Agent sees a state (from vision camera, and from 
the robot controller sensors); 

Step 2. The Agent takes actions (to drive any one or any 
combination of its joints and gripper); 

Step 3. The Agent receives a reward for each of its action; 

Step 4. Then the agent repeats from Step 1. 

We have the 
following table to match each step above to ML-programming.
\begin{table}[H] % place table here  
\renewcommand{\arraystretch}{1.3} 
\caption{Programming Aspects for ML}
\label{state-Table1}
\centering 
\begin{tabular}{|c||c||c|}
\hline
Category     & Description    & Note    \% Improvement \\
\hline
\hline
State & CollectObservations & Agent observes \\
\hline
Action(s) & OnActionReceived & Agent action \\
\hline
Reward & AddReward & Reward/Penalty \\
\hline
\hline 
\end{tabular}
\end{table}

\subsection{Creating ML Agent}
 
We create the Robot as 
“RobotControllerAgent.cs”. 
We perform initialization of 
the robot as follows, see the program line 32 to 37: 

\begin{verbatim}
public override void Initialize()
{
   ResetAllAxis();
   MoveToSafeRandomPosition();
   if (!trainingMode) MaxStep = 0;
}
\end{verbatim}

The robot is the ML Agent which 
should extend from class “Agent”, so we replace the default 
Monobehaviour to Agent in the script and start overriding the 
Agent methods as above initialization. 

\section{Mathematic Formulation}

Robotic operations can be characterized as a robot making
sequential movement, e.g., control actions in a stochastic 
environment. The control actions are torques from the
controllers to robot joints over a sequence of time steps. 
From the deep reinforcement learning point of view, the 
goal of robot operations is to maximize 
a long term reward. By the mathematical nature, these
sequential decision process is modeled as 
a Markov Decision Process (MDP). 

A large class of sequential decision 
making problems can be formuated as 
Markov decision processes (MDPs). 
An MDP consists of 

1. a set S of states, denoted as 
S = $\{s_1, s_2, ..., s_N \}$, 
and 

2. a set A of actions, denoted as 
A = $\{a_1, a_2, ..., a_M \}$ 
as well as 

3. a transition function
T and 

4. a reward function R, 

denoted as 
a tuple $< S, A, T, R >$ .
When in any state $s \in S$ , an action $a \in A$ 
will lead to a new state with a transition 
probability $P_T ( s, a, s\prime )$ , and 
a reward R( s, a ) function. 

The stochastic policy $\pi  : S \rightarrow D$ maps from a 
space state to a probability over the set of actions, 
and $\pi  ( a | s )$ represents the probability
of choosing action a at state s. 

The goal is to find the optimal policy
$\pi^*$ to produce the highest rewards  [Rein, 2020]:

\begin{equation} 
\arg\max_{\pi \in {\Omega}_{\pi}} 
E [\sum_{k=0}^{H-1} {\gamma}^k R (s_k, a_k) ] 
\} 
\end{equation}

The composition of the above equation can be 
described as follows. 
First, define reward function at time k: 

\begin{equation} 
R (s_k, a_k) 
\end{equation}

with discounted factor $\gamma$ at time $k$, and 
$\gamma < 1$ (to formulate the older state, as the k 
becomes bigger, its effects on the reward will get 
smaller): 

\begin{equation} 
{\gamma}^k R (s_k, a_k) 
\end{equation}   
Note In finite-horizon
or goal-oriented domains, choose discount factors close to 1 
to encourage actions towards the goal,
whereas in infinite-horizon domains choose lower discount factors
to achieve a balance between short- and long-
term rewards.

For all the experiments up to horizon H, we have summation: 
\begin{equation}    
\displaystyle\sum_{k=0}^{H-1} {\gamma}^k R (s_k, a_k) 
\end{equation}
 
For the stochastic nature of these rewards, we use
statistical expectation as

\begin{equation}  
E [\sum_{k=0}^{H-1} {\gamma}^k R (s_k, a_k) ]
=
\int_{\inf}  {\gamma}^k R (s_k, a_k) P(s_k, a_k) ds 
\end{equation}   

Hence, we have the average discounted 
rewards under policy $\pi$.    


\section{DRL Connection to Robot Control}

\textbf{Definition 1.} 
\textit{Trajectory $\tau$. 
A trajectory $\tau$ of a robot motion is defined as 
a sequence of state-action pairs in time sequence 
$t_1, t_2, ..., t_N$, denoted as 
$\tau = (s_1,a_1,s_2,a_2,...,s_N,a_N)$. 
} 
 
Consider any trajectory $\tau$ of a robot motion, e.g.,
$\tau = (s_1,a_1,s_2,a_2,...,s_N,a_N)$, 
the trajectory of the end effector in 3D space. 
See an end effector 
(a set of three in this case) from CTI's FD100 robot below, 



\textbf{Definition 2.} 
\textit{Reward $r$. 
A reward $r$ is defined as numerical value assigned to each 
state-action pair $s_i a_i$, e.g., formulate as 
$r: S \times A \rightarrow R$, where 
$S \times A $ = $(s_1 a_1, s_2 a_1, ..., s_1 a_N,$ 
$s_2 a_N, ......,s_N a_1,..., s_N a_N)$. 
} 
 

A reward $r$ is formulated as 
$r: S \times A \rightarrow R$, where 
$S \times A = (s_1 a_1, s_2 a_1, ..., s_1 a_N, s_2 a_N, ......,s_N a_1,..., s_N a_N)$.

We can build tables for S, A and R repectively, as follows 

\begin{table}[H] % place table here 
%% increase table row spacing, adjust to taste
\renewcommand{\arraystretch}{1.3}
% if using array.sty, it might be a good idea to tweak the value of
%\extrarowheight as needed to properly center the text within the cells
\caption{Table I. State Table for FD100 Robot}
\label{state-Table1}
\centering
%% Some packages, such as MDW tools, offer better commands for making tables
%% than the plain LaTeX2e tabular which is used here.
\begin{tabular}{|c||c||c|}
\hline
Category     & Description    & Note    \% Improvement \\
\hline
$s_{j1-angle}$     & $[,]$ & Continuous \\
\hline
$s_{j1-speed}$     & $[,]$ & Continuous \\
\hline 
$s_{j1-accel}$     & $[,]$ & Continuous \\
\hline 
$s_{j2-angle}$     & $[,]$ & Continuous \\
\hline
$s_{j2-speed}$     & $[,]$ & Continuous \\
\hline 
$s_{j2-accel}$     & $[,]$ & Continuous \\
\hline 
...     & ... & ... \\
\hline 
$s_{j6-angle}$     & $[,]$ & Continuous \\
\hline
$s_{j6-speed}$     & $[,]$ & Continuous \\
\hline 
$s_{j6-accel}$     & $[,]$ & Continuous \\
\hline 
\end{tabular}
\end{table}


\begin{table}[H] % place table here 
%% increase table row spacing, adjust to taste
\renewcommand{\arraystretch}{1.3}
% if using array.sty, it might be a good idea to tweak the value of
%\extrarowheight as needed to properly center the text within the cells
\caption{Table II. Acition Table for FD100 Robot}
\label{state-Table1}
\centering
%% Some packages, such as MDW tools, offer better commands for making tables
%% than the plain LaTeX2e tabular which is used here.
\begin{tabular}{|c||c||c|}
\hline
Category     & Description    & Note    \% Improvement \\
\hline
$a_{j1-angle}$     & $[,]$ & Continuous \\
\hline
$a_{j1-speed}$     & $[,]$ & Continuous \\
\hline 
$a_{j1-accel}$     & $[,]$ & ??? Check this \\
\hline 
$a_{j2-angle}$     & $[,]$ & Continuous \\
\hline
$a_{j2-speed}$     & $[,]$ & Continuous \\
\hline 
$a_{j2-accel}$     & $[,]$ & Continuous \\
\hline 
...     & ... & ... \\
\hline 
$a_{j6-angle}$     & $[,]$ & Continuous \\
\hline
$a_{j6-speed}$     & $[,]$ & Continuous \\
\hline 
$a_{j6-accel}$     & $[,]$ & ??? Check this \\
\hline 
\end{tabular}
\end{table}

\begin{table}[H] % place table here 
%% increase table row spacing, adjust to taste
\renewcommand{\arraystretch}{1.3}
% if using array.sty, it might be a good idea to tweak the value of
%\extrarowheight as needed to properly center the text within the cells
\caption{Table III. Reward Table for FD100 Robot}
\label{state-Table1}
\centering
%% Some packages, such as MDW tools, offer better commands for making tables
%% than the plain LaTeX2e tabular which is used here.
\begin{tabular}{|c||c||c|}
\hline
Category     & Description    & Note    \% Improvement \\
\hline
$s_{j1-angle} \times a_{j1-angle}$     & $[,]$ & Continuous \\
\hline
$s_{j1-speed} \times a_{j1-angle}$    & $[,]$ & Continuous \\
\hline 
$s_{j1-accel} \times a_{j1-angle}$     & $[,]$ & Continuous \\
\hline 
...     & ... & ... \\
\hline 
\end{tabular}
\end{table}

Note, check the datasheet of FD100 to fill in the numerical values in 
the description section of the above tables. 

\section{Policy On Robot Operations}

\textbf{Definition 3.} 
\textit{Policy $\pi$.
A policy $\pi$ in Robotics is a set of guidelines for 
a robot controller to follow to deliver its control action $a_i$ upon its 
current state $s_i$, which is denoted as $\pi (a_i |\ s_i)$. 
} 
  
  
A policy $\pi$ leads to the robot control to deliver
its control action as a mapping function $s_i \rightarrow a_i$. 

A policy $pi$ can be either stochastic which is characterized by a 
conditional probability as

\begin{equation}
\pi(a_i | s_i) : s_i \rightarrow Pr(a_i | s_i), 
\end{equation} 

or deterministic  

\begin{equation}
\pi(a_i | s_i) : s_i \rightarrow a_i = \mu (s_i).  
\end{equation} 

\section{Policy As DNN}

We now introduce a notation to policy $\pi$ as 
$\pi_{\theta}$ where a set of variables which affects $\pi .$
In deep reinforcement learning (DRL), a policy 
$\pi_{\theta}$ 
is formulated as a deep neural network
(DNN), where $\theta$ is the general parameter storing
all the network’s weights and biases $W=(w_{i,j})$.

\textbf{Definition 4.} 
\textit{Policy $\pi_{\theta}$.    
A policy $\pi_{\theta}$ is a deep neural network
(DNN), where $\theta$ is the collection of all parameters  
of the neural network’s (NN) weights and biases, simply 
denoted as $W=(w_{i,j})$. 
} 
 
A typical realization of $\pi_{\theta}$ is a
gaussian Multi-Layer Perceptron (MLP) net, which samples
the action to be taken from a gaussian distribution of actions
over states as follows 
 
 
\section{Quick Review}
 
The objectives of deep reinforcement learning is to 
develop algorithms to make robot has the ability to learn new 
manipulation policies 

(1) from scratch, without user
demonstrations and 

(2) without the need of a task-specific 
domain knowledge. 
 
\subsection{Trust Region Policy Optimization} 
Trust Region Policy Optimization (TRPO) [1]
 

\subsection{Deep Q-Network} 
 Deep Q-Network with Normalized Advantage Func-
tions (DQN-NAF) [ 


\subsection{Deep Deterministic Policy Gradient} 
Deep Deterministic Policy Gradient
(DDPG) [3] and 


\subsection{Vanilla Policy Gradient} 

Vanilla Policy Gradient (VPG) [4].  


\subsection{hyper-parameters selection and tuning procedures} 

The hyper-parameters selection and
tuning procedures



 as well as demonstrate the robustness
and adaptability of TRPO and DQN-NAF while performing
manipulation tasks such as reaching a random position target
and pick \& placing an object. 






 Moreover, their model-freedom guarantees
good performances even in case of changes in the dynamic
and geometric models of the robot (e.g., link lengths, masses,
and inertia).
 
 ???

   
\section*{Acknowledgment}

I would like to express my thanks to CTI One engineering member and 
CTI One engineering intern team, as well as to SJSU graduate research assistants 
for letting me introducing DRL in Robotics to our next project and for the 
codings in Deep Learning, Robotis and embedded systems. 
\begin{thebibliography}{1}

\bibitem{Franceschetti, 2020}
[Franceschetti, 2020] 
Andrea Franceschetti, Elisa Tosello, Nicola Castaman, and Stefano
Ghidoni, 
"Robotic Arm Control and Task Training
through Deep Reinforcement Learning", 
https://arxiv.org/pdf/2005.02632.pdf, May 2020. 

\bibitem{Reinforcement, 2020} 
[Reinforcement, 2020] 
Reinforcement learning, 
https://en.wikipedia.org/wiki/Reinforcement \\ \_learning, 2020. 

\end{thebibliography}
 

% that's all folks
\end{document}


