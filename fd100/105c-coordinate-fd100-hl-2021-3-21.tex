\documentclass[conference]{IEEEtran} 

%2019 hl ref: 
%https://en.wikibooks.org/wiki/LaTeX/Mathematics#Matrices_and_arrays
\ifCLASSINFOpdf 
\else 
\fi  
\hyphenation{op-tical net-works semi-conduc-tor}

%\usepackage{amsmath} 
%HL 2018-1-21: not sure if IEEE accepts this

\usepackage{graphicx}
\graphicspath{images/} %HL 2018-1-21 add this image folder 
\usepackage{float} % HL to force latex figure location

\voffset 5 pt %HL 2021-3-20 force vertical offset at each page

\begin{document}

\title{{\small Engineering Application Note 105cFD100 6 DoF Robot}
\\Robotics Coordinate System  
}
 
\author{
\IEEEauthorblockN{Harry Li $^{\ddagger}$, Ph.D.}
\IEEEauthorblockA{ 
Computer Engineering Department, San Jose State University\\
San Jose, CA 95192, USA} \\
Email$^{\dagger}$: harry.li@ctione.com\\
}

% make the title area
\maketitle
\begin{abstract}
This note describes the coordinate systems for FD100
robotics system.
\end{abstract}

% no keywords 
\IEEEpeerreviewmaketitle

\section{Introduction}

This note describes the coordinate systems for FD100
robotics system. The reference material for 
this discussion is from the following document:  

1. CTI One's Technical Document (PPT)  
\begin{verbatim}
https://github.com/hualili/robotics-
open_abb/blob/master/
fd100/2021S-104-%232019%236-20-Turin-
Robotics-RemoteControl-v1.0.pdf
\end{verbatim} 

2. Turin Smart Robot General Operation Manual V2.0.pdf 

3. Harry Li's deep reinforcement learning white paper
(Part I on Deep Reinforcement Learning); 

4. Robot Arm training github repository, 
https://github.com/rkandas/RobotArmMLAgentUnity; 

5. Published paper by Google Deepmind team. 

6. Tutorial on 6 DoF with ML model. 
\begin{verbatim}
https://medium.com/xrpractices/how-
to-train-your-robot-arm-fbf5dcd807e1
\end{verbatim}

By manufacturer's product default definition, 
robot supports 4 coordinate systems [2],
listed in the following table. In order to 
integrate Computer Vision capability, we have 
also added the additional coordinate systems, e.g., the 
viewer coordinate system to form the new capability. 
\begin{table}[H] % place table here  
\renewcommand{\arraystretch}{1.3} 
\caption{Coordinate Systems}
\label{state-Table1}
\centering 
\begin{tabular}{|c||c|}
\hline
Category         & Description   \\
\hline
\hline
Joint Coordinate JCS & $j_1,...,j_6$ \\
\hline
Rectangular Coordinate WCS & $x_w-y_w-z_w$ \\
\hline
User Coordinate UCS & $x_u-y_u-z_u$ \\
\hline
Tool Coordinate TCS & $x_t-y_t-z_t$ \\
\hline
\hline 
Viewer Coordinate VCS & $x_v-y_v-z_v$ \\
\hline
\end{tabular}
\end{table}


1. The Joint Coordinate System (JCS) $j_1,...,j_6$
is defined to describe each axis 
angular position.  

2. The Rectangular Coordinate System is the coordinate system 
whose origin is defined at the centre of the robot
base and it servers as the base line reference system 
to establish the relationship with the rest of the 
coordinate system. Therefore it is also called 
world coordinate system (WCS) $x_w-y_w-z_w$. 

3. The Tool Coordinate System (TCS) $x_t-y_t-z_t$  
is defined on the fixture of wrist flange plate of the robot, and
it can be modifed and defined by users. The effective direction 
$d_t$ of the fixture is defined as the z axis $z_t$ of the tool coordinate system,
$d_t = z_t$. 

4. The User Coordinate System (UCS) is defined on the work piece 
reacheable by the robot end effector, and is defined by users.

5. The viewer coordinate system (VCS) is defined to describe the 
images and videos captured on a given camera which (1) it 
looks at work space defined UCS, and (2) it looks at the 
origin of WCS, ideally, as more operational tasks added to 
the robot system, the 2nd, 3rd VCS may be added, in this case, 
we would have $VCS_i$ for i more than 1. 

All the experiments will be carried out in FD100
robotics system shown in this figure. 
\DeclareGraphicsExtensions{.eps} %.eps
%\begin{figure}[Htb] 
\begin{figure}[H] 
\centering
\includegraphics[scale = 0.35]{figure2}  
\caption{All the experiments will be carried out in FD100
robotics system.}
\label{figure2} 
\end{figure}

\section{World (Rectangular) Coordinate System}
The world (also known as rectangular) 
coordinate system is the right hand (RH)
coordinate system, denoted as $x_w-y_w-z_w$, with its 
origin at the center of the FD100 robot as shown in 
the figure below. 
\DeclareGraphicsExtensions{.eps} %.eps
%\begin{figure}[Htb] 
\begin{figure}[H] 
\centering
\includegraphics[scale = 0.35]{figure1-world-coordinate}  
\caption{The world coordinate system $x_w-y_w-z_w$ with its 
origin at the center of the FD100.}
\label{figure1-world-coordinate} 
\end{figure} 
The relationship between the world coordinate system 
and the robot joints is illustrated below. 
\DeclareGraphicsExtensions{.eps} %.eps
%\begin{figure}[Htb] 
\begin{figure}[H] 
\centering
\includegraphics[scale = 0.35]{figure1-6dof-robot} %name of the eps figure file 
\caption{The robot joints diagram.}
\label{figure1-6dof-robot} 
\end{figure}
The figure is from the Turin Robot User Guide, pp. 23.  


\section{User Coordinate System}
An user coordinate system UCS is RH system denoted 
as $x_u-y_u-z_u$ which is defined
by user to describe the work space which the robot
end effector can reach. 

According to the technical report [1] and [2], to set an 
user coordinate system, we use 3-point method which 
define 1 point at the origin, and 1 point on the positive $x_u$
axis, and 1 point on the $x_u-y_u$ plane. The operation
steps for this is described below according to reference
[1] and [2]: 
 
Step 1. Move the robot with the operating tool to 
the origin of the (user) coordinate system and record 
the point.

Step 2. Move the robot with the operating tool to 
any point on the x-axis and record the point.

Step 3. Move the robot with the operating tool to any 
point in the xy plane that belongs to the first 
quadrant and record the point.

Step 4. Save the calculation results. Hence the user
coordinate system is established. 

\DeclareGraphicsExtensions{.eps} %.eps
%\begin{figure}[Htb] 
\begin{figure}[H] 
\centering
\includegraphics[scale = 0.35]{figure1-user-coordinate}  
\caption{An user coordinate system $x_u-y_u-z_u$ 
to describe the work space.}
\label{figure1-user-coordinate} 
\end{figure}

\section{Tool Coordinate System}
A tool coordinate system TCS is denoted 
as $x_t-y_t-z_t$ which is defined
by user to operate the robot
end effector for its operation. This coordinate system 
requires a process to set it up by the user. 
This Coordinate system is defined on the tool.   
The effective direction of tool is defined as the $z_t$ axis, 
and the $x_t$ axis and $y_t$ axis are defined according to 
the right-hand rule.
\DeclareGraphicsExtensions{.eps} %.eps
%\begin{figure}[Htb] 
\begin{figure}[H] 
\centering
\includegraphics[scale = 0.35]{figure1-tool-coordinate}  
\caption{An user coordinate system $x_u-y_u-z_u$ 
to describe the work space.}
\label{figure1-tool-coordinate} 
\end{figure}

Set up TCS take 6 steps described here based on reference 
[6]. Illustrated in the figure here is 6 different tool 
points. The user will have to move the robot to each of 
the point and record it. 
\DeclareGraphicsExtensions{.eps} %.eps
%\begin{figure}[Htb] 
\begin{figure}[H] 
\centering
\includegraphics[scale = 0.35]{figure2-tool-coordinate}  
\caption{Illustration of 6-point TCS setup process.}
\label{figure2-tool-coordinate} 
\end{figure}

The set up process of TCS is described in the reference
document [6] 

\DeclareGraphicsExtensions{.eps} %.eps
%\begin{figure}[Htb] 
\begin{figure}[H] 
\centering
\includegraphics[scale = 0.35]{figure3-tool-coordinate}  
\caption{The GUI from robot teach pandant for setting up TCS.}
\label{figure3-tool-coordinate} 
\end{figure}

\section{Experiment Set Up to Test User Coordinate}
At CTI One's robotics lab, an experiment setup
allows FD100 robot to test its user space with the 
computer vision system's imaging data $I_v(x,y)$ 
which is defined in viewer coordinate system 
$x_v-y_v-z_v$. 
\DeclareGraphicsExtensions{.eps} %.eps
%\begin{figure}[Htb] 
\begin{figure}[H] 
\centering
\includegraphics[scale = 0.25]{figure2-user-coordinate}  
\caption{Test an user space with the 
computer vision system's imaging data $I_v(x,y)$ 
in the viewer coordinate system 
$x_v-y_v-z_v$ .}
\label{figure2-user-coordinate} 
\end{figure}

To prepare for the Computer Vision guided robot experiment, 
build a checker board and set the checker board in 
a visible location for video camera $CAM_1$ and to allow robot 
end effector reachable 
to any of the point on the board, as illustrated in the 
UCS photo in this section. 

Computer Vision system plays a significant role in 
robotics operation: 

1. Computer Vision system with deep learning capability can 
identify an object. 

2. Then a mapping from a viewer coordinate system (VCS) $x_v-y_v-z_v$
to world coordinate system (WCS) $x_w-y_w-z_w$ allows the 
object location to be calculated in the WCS.  

3. Compute the 
inverse kinematics. Note, inverse kinematics is the mathematical 
process of computing the joint parameters needed to place the end 
effector of a kinematic chain, e.g., a robot manipulator
in a given position and orientation relative to the start of the 
chain. 

4. Move the robot to the joint parameters based on certain
rules or guidelines, such as, best long term reward, to reach and 
grip the object. 

Hence, the mathematical forumulation for the mapping 
from a viewer coordinate system (VCS) $x_v-y_v-z_v$
to world coordinate system (WCS) $x_w-y_w-z_w$ is crucial 
step to complete this computer vision based automated 
process. We just cover the basic introduction here, 
we will give detailed discussion on this technique
and its realization in a separate engineering note.  

\begin{figure}[H] 
\centering
\includegraphics[scale = 0.35]{figure1-mapping}  
\caption{Mapping with rotation and translation.}
\label{figure1-mapping} 
\end{figure}

\begin{figure}[H] 
\centering
\includegraphics[scale = 0.35]{figure2-mapping}  
\caption{Mapping example on a checker board with rotation and translation.}
\label{figure12-mapping} 
\end{figure}

Mathematical analysis leads to a homography matrix H. 
The homography matrix H relates the positions of the point(s)
$P_{o}$ on a source image plane
to the points on the destination imager plane $P_{v}$ by 
the following equation

\begin{equation}
P_{o}(x_o,y_o,z_o) = H^{-1} P_v(x_v,y_v) . 
\end{equation}
 
OpenCV provides cvFindHomography() function, which takes a list of
correspondences and returns the homography matrix H that best describes those 
correspondences. 
A minimum of four points to solve for H is needed, but many
more would be better as we can solve over-determined 
system. In the case of chessboard bigger than 3-by-3, 
more matching pair of points is not a problem. Using more
points is better for noise reduction. 

\section{Robot Movement}
 
FD100 robot has 3 different types of movement listed in the 
following table. 
\begin{table}[H] % place table here  
\renewcommand{\arraystretch}{1.3} 
\caption{Three Type of Movement}
\label{Table1-movement}
\centering 
\begin{tabular}{|c||c|}
\hline
Category         & Description   \\
\hline
\hline
Joint movement  & MOVJ \\
\hline
Linear movement & MOVL \\
\hline
Arc movement    & MOVC \\
\hline
\hline
\end{tabular}
\end{table}
 
Joint movement is the movement defined by the joint angular 
change. 

Linear movement is the movement to move the robot to the present (demonstration) 
point through linear path. During the movement, the
robot’s movement control point shall follow straight line, and the gesture of 
the fixture shall be
changed automatically as shown in the following figure. 

\DeclareGraphicsExtensions{.eps} %.eps
%\begin{figure}[Htb] 
\begin{figure}[H] 
\centering
\includegraphics[scale = 0.25]{figure1-linear-movement}  
\caption{Illustration of linear movement with the gestrure of fixture.}
\label{figure1-linear-movement} 
\end{figure}

Arch movement is needed when 
the robot needs to move to the present demonstration point through arc path. 
Three points determine an unique arc, so three points are needed to 
define an arc movement.
Arc movement is realized by starting from the present point, passing 
through the first point (auxiliary
point) and reaching the second point (ending point).  
 
\section*{Acknowledgment}

I would like to express my thanks to CTI One engineering member and 
CTI One engineering intern team, as well as to SJSU graduate research assistants 
for letting me introducing DRL in Robotics to our next project and for the 
codings in Deep Learning, Robotis and embedded systems. 

\section*{Quiz}

1. How many coordinate systems have we defined in this note? list each of them
and explain its definition? 

2. Draw illustration of JCS, WCS, UCS, TCS, VCS? 

3. What is homography matrix H? how many corresponding pair of points needed
to define H? 

4. Suppose H matrix is found, then how would you find  
$P_{o}$ on a source image plane? 
(Answer: use 
$P_{o}(x_o,y_o,z_o) = H^{-1} P_v(x_v,y_v)$.)

5. How many type of movement? name each of them, draw 3D motion trajectory 
to illustrate each of the movement. 

6. Design a peak and place operation for a robot to pick a coffee cup on
a table top and place it back on the same table top but at different 
location by listings each of the movement type, (assuming the robot start
from an initial position $(x_t,y_t,z_t)$. 
 
\begin{thebibliography}{1} 
\bibitem{Franceschetti, 2020}
[Franceschetti, 2020] 
Andrea Franceschetti, Elisa Tosello, Nicola Castaman, and Stefano
Ghidoni, 
"Robotic Arm Control and Task Training
through Deep Reinforcement Learning", 
https://arxiv.org/pdf/2005.02632.pdf, May 2020. 

\bibitem{Reinforcement, 2020} 
[Reinforcement, 2020] 
Reinforcement learning, 
https://en.wikipedia.org/wiki/Reinforcement \\ \_learning, 2020. 

\bibitem{OpenCV, 2017} 
[OpenCV, 2017] 
Learning OpenCV, Book, 2017. 

\end{thebibliography}
 

% that's all folks
\end{document}


