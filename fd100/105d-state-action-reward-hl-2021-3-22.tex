\documentclass[conference]{IEEEtran} 

%2019 hl ref: https://en.wikibooks.org/wiki/LaTeX/Mathematics#Matrices_and_arrays
\ifCLASSINFOpdf 
\else 
\fi  
\hyphenation{op-tical net-works semi-conduc-tor}

%\usepackage{amsmath} 
%HL 2018-1-21: not sure if IEEE accepts this

\usepackage{graphicx}
\graphicspath{images/} %HL 2018-1-21 add this image folder 
\usepackage{float} % HL to force latex figure location

%HL 2020-10-19 add this for pseudo code 
\usepackage{algorithm}
\usepackage{algorithmic}

%HL 2021-3-22 for highlighting the text
\usepackage{xcolor, soul}
\sethlcolor{yellow}

\begin{document}
 
% paper title 

\title{{\small Engineering Application Note 105dFD100 Deep Reinforcement Learning} \\
States And Actions As Well As Rewards
}
 
\author{
\IEEEauthorblockN{Harry Li $^{\ddagger}$, Ph.D.}
\IEEEauthorblockA{ 
Computer Engineering Department, San Jose State University\\
San Jose, CA 95192, USA} \\
Email$^{\dagger}$: harry.li@ctione.com\\
}

% make the title area
\maketitle
\begin{abstract}
This note describes the states, actions, and rewards
based on Unity AI model as reference guide for 
deep reinforcement learning experiment design.   
\end{abstract}

% no keywords 
\IEEEpeerreviewmaketitle

\section{Introduction}

This note is prepared based on the following material: 

1. nnn-n-6DoF-Action-State-Reward-SS-2021-03-17.odt;

2. readme for Unity AI robotics training. 

Robotic operations can be characterized as a robot making
sequential movement, e.g., control actions in a stochastic 
environment. By the mathematical nature, these
sequential decision process is modeled as 
a Markov Decision Process (MDP) which consists of 

1. a set S of states, denoted as 
S = $\{s_1, s_2, ..., s_N \}$, 
and 

2. a set A of actions, denoted as 
A = $\{a_1, a_2, ..., a_M \}$ 
as well as  

3. a transition function
T and 

4. a reward function R, 

denoted as 
a tuple $< S, A, T, R >$ .
When in any state $s \in S$ , an action $a \in A$ 
will lead to a new state with a transition 
probability $P_T ( s, a, s\prime )$ , and 
a reward R( s, a ) function. 

The stochastic policy $\pi  : S \rightarrow D$ maps from a 
space state to a probability over the set of actions, 
and $\pi  ( a | s )$ represents the probability
of choosing action a at state s. 

The goal is to find the optimal policy
$\pi^*$ to produce the highest rewards  [Rein, 2020]:

\begin{equation} 
\arg\max_{\pi \in {\Omega}_{\pi}} 
\{ 
E 
[ \displaystyle\sum_{k=0}^{H-1} {\gamma}^k R (s_k, a_k) ] 
\} 
\end{equation}

For the stochastic nature of these rewards, we use
statistical expectation as

\begin{equation}  
E [ \sum_{k=0}^{H-1} {\gamma}^k R (s_k, a_k) ]  
=
\int_{\inf}  {\gamma}^k R (s_k, a_k) P(s_k, a_k) ds 
\end{equation}   
Hence, we have the average discounted 
rewards under policy $\pi$.    

\section{States and Actions}
\textbf{Definition 1.} 
\textit{Trajectory $\tau$. 
A trajectory $\tau$ of a robot motion is defined as 
a sequence of state-action pairs in time sequence 
$t_1, t_2, ..., t_N$, denoted as 
$\tau = (s_1,a_1,s_2,a_2,...,s_N,a_N)$. 
} 
 
Consider any trajectory $\tau$ of a robot motion, e.g.,
$\tau = (s_1,a_1,s_2,a_2,...,s_N,a_N)$, 
the trajectory of the end effector in 3D space. 
See an end effector 
(a set of three in this case) from CTI's FD100 robot below, 

\DeclareGraphicsExtensions{.eps} %.eps
%\begin{figure}[Htb] 
\begin{figure}[H] 
\centering
\includegraphics[scale = 0.35]{figure1} %name of the eps figure file 
\caption{An end effector 
(a set of three in this case) from CTI's FD100 robot below, whose
movement forms a trajectory in 3D space". }
\label{fig1} 
\end{figure} 
 
The bigger view of the FD100 robot with an end effector 
(a set of three in this case) in the view. 
\DeclareGraphicsExtensions{.eps} %.eps
%\begin{figure}[Htb] 
\begin{figure}[H] 
\centering
\includegraphics[scale = 0.35]{figure2} %name of the eps figure file 
\caption{The bigger view of the FD100 robot with an end effector 
(a set of three in this case) in the view.}
\label{fig2} 
\end{figure}

\section{Reward}
\textbf{Definition 2.} 
\textit{Reward $r$. 
A reward $r$ is defined as numerical value assigned to each 
state-action pair $s_i a_i$, e.g., formulate as 
$r: S \times A \rightarrow R$, where 
$S \times A $ = $(s_1 a_1, s_2 a_1, ..., s_1 a_N,$ 
$s_2 a_N, ......,s_N a_1,..., s_N a_N)$. 
} 
 
We can build tables for S, A and R repectively, as follows 

\begin{table}[H] 
\renewcommand{\arraystretch}{1.3} 
\caption{Table I. State Table from Unity AI Robot}
\label{state-Table1}
\centering 
\begin{tabular}{|c||c||c|}
\hline
Category     & Description    & Note    \% Improvement \\
\hline
$s_{j1-angle}$     & $[,]$ & Continuous \\
\hline
$s_{j1-speed}$     & $[,]$ & Continuous \\
\hline 
$s_{j1-accel}$     & $[,]$ & Continuous \\
\hline 
$s_{j2-angle}$     & $[,]$ & Continuous \\
\hline
$s_{j2-speed}$     & $[,]$ & Continuous \\
\hline 
$s_{j2-accel}$     & $[,]$ & Continuous \\
\hline 
...     & ... & ... \\
\hline 
$s_{j6-angle}$     & $[,]$ & Continuous \\
\hline
$s_{j6-speed}$     & $[,]$ & Continuous \\
\hline 
$s_{j6-accel}$     & $[,]$ & Continuous \\
\hline 
\end{tabular}
\end{table}


\begin{table}[H]  
\renewcommand{\arraystretch}{1.3} 
\caption{Table II. Acition Table From Unity AI Robot}
\label{state-Table1}
\centering 
\begin{tabular}{|c||c||c|}
\hline
Category     & Description    & Note    \% Improvement \\
\hline
$a_{j1-angle}$     & $[,]$ & Continuous \\
\hline
$a_{j1-speed}$     & $[,]$ & Continuous \\
\hline 
$a_{j1-accel}$     & $[,]$ & ??? Check this \\
\hline 
$a_{j2-angle}$     & $[,]$ & Continuous \\
\hline
$a_{j2-speed}$     & $[,]$ & Continuous \\
\hline 
$a_{j2-accel}$     & $[,]$ & Continuous \\
\hline 
...     & ... & ... \\
\hline 
$a_{j6-angle}$     & $[,]$ & Continuous \\
\hline
$a_{j6-speed}$     & $[,]$ & Continuous \\
\hline 
$a_{j6-accel}$     & $[,]$ & ??? Check this \\
\hline 
\end{tabular}
\end{table}

The design of reward function $R(s_t, a_t)$ is defined 
based on the general guidelines from Unity AI Robot github [???] 

1. When the arm hits the ground, a Hefty Penalty (-1) is 
given, e.g., 

\begin{equation}
R(s_t,a_t) = -1
\end{equation}
where $s_t =$ (ground state),  
then the training episode is terminated. 
Note each pisode is defined as one full cycle of training. 

2. When the arm reaches the target, a Hefty Reward (1) 
is given,  
\begin{equation}
R(s_t,a_t) = 1
\end{equation}
where $s_t =$ (target state). The 
target state can be detected when the end effector 
$P_{end}(x,y,z)$ 
reaches the object(target) $P_{tgt}(x,y,z)$, e.g., 

\begin{equation}
|| P_{end}(x,y,z) - P_{tgt}(x,y,z) || \leq \epsilon
\end{equation}

Then end the episode;
 
3. When the arm reaches closer to the target, a
marginal reward is assigned based on the normalized  
difference in distance to the targe. That is to define
the previous arm's (end effector) position as 
$P_{end}(x(t-1),y(t-1),z(t-1))$, if the current position
is $P_{end}(x(t),y(t),z(t))$, then the distances to the 
target position can be defined as 
\begin{equation}
d(P_{end}(t-1),P_{tgt}) = || P_{end}(t-1) - P_{tgt}  ||_2 
\end{equation} 
and 
\begin{equation}
d(P_{end}(t),P_{tgt}) = || P_{end}(t) - P_{tgt}  ||_2 
\end{equation}
if we have 
\begin{equation}
d(P_{end}(t),P_{tgt}) \leq d(P_{end}(t-1),P_{tgt}) 
\end{equation}
then, the reward is defined as the distance gained to the 
target 
\begin{equation}
R(s_t,a_t) = \frac{ d(P_{end}(t),P_{tgt}) - d(P_{end}(t-1),P_{tgt})}
{ d(P_{end}(t),P_{tgt}) }  
\end{equation}
Note, 
\begin{equation}
1 \geq R(s_t,a_t) \geq 0. 
\end{equation}
4. When the arm moves far from the target, based on the similar
notion for in 3, we define a reward as a marginal penalty
as how far is it moved away from the target as: 

if
\begin{equation}
d(P_{end}(t),P_{tgt}) \geq d(P_{end}(t-1),P_{tgt}) 
\end{equation}
\begin{equation}
R(s_t,a_t) = \frac{ d(P_{end}(t),P_{tgt}) - d(P_{end}(t-1),P_{tgt})}
{ d(P_{end}(t),P_{tgt}) }
\end{equation}
Note, 
\begin{equation}
-1 \leq R(s_t,a_t) \leq 0. 
\end{equation}
\begin{table}[H] % place table here 
\renewcommand{\arraystretch}{1.3} 
\caption{Table III. Reward Table}
\label{state-Table1}
\centering 
\begin{tabular}{|c||c||c|}
\hline
Category     & Description    & Note    \% Improvement \\
\hline
$s_{j1-angle} \times a_{j1-angle}$     & $[,]$ & Continuous \\
\hline
$s_{j1-speed} \times a_{j1-angle}$    & $[,]$ & Continuous \\
\hline 
$s_{j1-accel} \times a_{j1-angle}$     & $[,]$ & Continuous \\
\hline 
...     & ... & ... \\
\hline 
\end{tabular}
\end{table}

Note, check the datasheet of FD100 to fill in the numerical values in 
the description section of the above tables. 

\section{Policy On Robot Operations \hl{(Under Construction)}}
 
\textbf{Definition 3.} 
\textit{Policy $\pi$.
A policy $\pi$ in Robotics is a set of guidelines for 
a robot controller to follow to deliver its control action $a_i$ upon its 
current state $s_i$, which is denoted as $\pi (a_i |\ s_i)$. 
} 
  
A policy $\pi$ leads to the robot control to deliver
its control action as a mapping function $s_i \rightarrow a_i$. 

A policy $pi$ can be either stochastic which is characterized by a 
conditional probability as

\begin{equation}
\pi(a_i | s_i) : s_i \rightarrow Pr(a_i | s_i), 
\end{equation} 

or deterministic  

\begin{equation}
\pi(a_i | s_i) : s_i \rightarrow a_i = \mu (s_i).  
\end{equation} 

\section{Policy As DNN \hl{(Under Construction)}}

We now introduce a notation to policy $\pi$ as 
$\pi_{\theta}$ where a set of variables which affects $\pi .$
In deep reinforcement learning (DRL), a policy 
$\pi_{\theta}$ 
is formulated as a deep neural network
(DNN), where $\theta$ is the general parameter storing
all the network’s weights and biases $W=(w_{i,j})$.

\textbf{Definition 4.} 
\textit{Policy $\pi_{\theta}$.    
A policy $\pi_{\theta}$ is a deep neural network
(DNN), where $\theta$ is the collection of all parameters  
of the neural network’s (NN) weights and biases, simply 
denoted as $W=(w_{i,j})$. 
} 
 
A typical realization of $\pi_{\theta}$ is a
gaussian Multi-Layer Perceptron (MLP) net, which samples
the action to be taken from a gaussian distribution of actions
over states as follows 

\begin{equation}
\pi_{\theta} ( a_i | s_i ) = 
\frac{ 1} { \sqrt{ (2 \pi)^{n_a} det \sum_{\theta}(s) } }   
E [ - \frac{ ( {||a - \mu_{\theta}(s)|| )^2} } { 2  \sum_{\theta}(s)} ]
\end{equation} 

\textbf{Example 1.}  
Supppose we have $s_1 = -1, s_2 = 2$
and $a_1 = 10$ and $a_2 = 20$, 
find

(1) $\sum_{\theta}(s)$ and $det \sum_{\theta}(s)$, 

(2) $ \mu_{\theta}(s)$, 

(3) $\pi_{\theta} ( a_i | s_i ) =?$ 

Sol: 

(1) $\sum_{\theta}(s) = \frac{1} {2}  (s_1 + s_2) $, 
 
\section*{Acknowledgment}

I would like to express my thanks to CTI One engineering member and 
CTI One engineering intern team, as well as to SJSU graduate research assistants 
for letting me introducing DRL in Robotics to our next project and for the 
codings in Deep Learning, Robotis and embedded systems. 

\section{Quiz}

1. Suppose the target position is $P_{tgt}$ = $(110,110,200)$, 
the robot end effector previous postion is 
$P_{end}(x(t-1),y(t-1),z(t-1))$ = $(11.35, 113.6, 201.5)$, 
the current position
is $P_{end}(x(t),y(t),z(t))$ =$(23.1,114,203.1)$, 
find the reward function $R(s_t,a_t)$=? 


\begin{thebibliography}{1}
\bibitem{Franceschetti, 2020}
[Franceschetti, 2020] 
Andrea Franceschetti, Elisa Tosello, Nicola Castaman, and Stefano
Ghidoni, 
"Robotic Arm Control and Task Training
through Deep Reinforcement Learning", 
https://arxiv.org/pdf/2005.02632.pdf, May 2020. 

\bibitem{Reinforcement, 2020} 
[Reinforcement, 2020] 
Reinforcement learning, 
https://en.wikipedia.org/wiki/Reinforcement \\ \_learning, 2020. 

\end{thebibliography}
\end{document}


