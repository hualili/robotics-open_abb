\documentclass[conference]{IEEEtran} 

%2019 hl ref: https://en.wikibooks.org/wiki/LaTeX/Mathematics#Matrices_and_arrays
\ifCLASSINFOpdf 
\else 
\fi  
\hyphenation{op-tical net-works semi-conduc-tor}

%\usepackage{amsmath} 
%HL 2018-1-21: not sure if IEEE accepts this

\usepackage{graphicx}
\graphicspath{images/} %HL 2018-1-21 add this image folder 
\usepackage{float} % HL to force latex figure location

%HL 2020-10-19 add this for pseudo code 
\usepackage{algorithm}
\usepackage{algorithmic}

%HL 2021-3-22 for highlighting the text
\usepackage{xcolor, soul}
\sethlcolor{yellow}

%HL 2021-6-1 for curly bracket equation
\usepackage{amsmath} 

\begin{document}
 
% paper title 

\title{ 
Accelerated Reward Policy (ARP) For Robotics Deep Reinforcement Learning  
}
 
\author{
\IEEEauthorblockN{Harry Li, Chee Vang, Shifabanu Shaikh, 
Yusuke Yakuwa $^{\dagger}$, Allen Lee  $^{\dagger}$ \\
Nitin Patil $^{\dagger}$, Nisarg Vadher, 
Zhixuan Zhou $^{\dagger}$, Shuwen Zheng $^{\dagger}$}
\IEEEauthorblockA{ 
Computer Engineering Department, San Jose State University, 
San Jose, CA 95192, USA \\
$^{\dagger}$ CTI One Corporation, Santa Clara, CA 95051, 
hua.li@sjsu.edu, $^{\dagger}$ yusuke.yakuwa.ctione.com, 
$^{\dagger}$ allenlee@usc.edu
\\
}} 

% make the title area
\maketitle
\begin{abstract}
Reward policy is a crucial part for Deep Reinforcement Learning (DRL)
applications in Robotics. The challenges for autonomous systems
with "human-like" behavior have posed significant need for a better,
faster, and more robust training based on optimized reward function. 
Inspired by the Berkeley and Google's work, this paper addresses
our recent development in reward policy/function design. In particular, 
we have formualated an accelerated reward policy (ARP) based on 
a non-linear functions. We have applied this reward function 
to SAC (Soft Actor Critic) algorithm for 6 DoF (Degree of Freedom)
robot training in simulated environment using Unity Gaming platform
and a 6 DoF robot. 
This nonlinear ARP  
function gives bigger reward to accelerate the robot's positive 
behavior during the trainning. 
Comparing to the existing algorithm our experimental 
results demonstrated faster convergence and bigger, better  
accumulative reward. With limited experimental data, the results
show improved accumulative reward function as much as 2 times of 
the previous results.\footnote{This paper has been submitted to
the International Conference: the Future of Information and Communication 
Conference (FICC), San Francisco, March 2022}  
\end{abstract}

% no keywords 
\IEEEpeerreviewmaketitle

\section{Introduction}
The challenges for autonomous systems
with "human-like" behavior have posed significant need for a better,
faster, and more robust training based on optimized reward function. 
Inspired by the work [3,4,5,6,18,19,20] and [1,8].

Reward policy is a cruicial part for Deep Reinforcement Learning (DRL)
applications in Robotics. 
This paper addresses
our recent development in reward policy/function design. 
In particular, 
we have formualted an accelerated reward policy (ARP) based on 
a non-linear function. We have applied this reward function 
to SAC (Soft Actor Critic) algorithm 
[3,4,5,6] 
for 6 DoF (Degree of Freedom)
robot training in simulated environment using Unity Gaming platform. 
This nonlinear ARP  
function gives bigger reward to accelerate the robot's positive 
behavior during the trainning.

The soft actor critic algorithm (SAC) [3] is the technique 
widely adoption for deep reinforcement learning (DRL).
The scope of this algritm is illustrated in Figure 1. Note
that:  

\begin{itemize}
\item[1] A Markov Decision Process (MDP) which serves 
as the foundation for the discussion of all algorithms here; 

\item[2] Deep-Q (DQ) and double Deep-Q (DDQ) algorithms 
for the discrete control actions based DRL; and 

\item[3] Stochastic policy gradient (SPG) algorithm and 
SAC algorithm for continuous control action 
based DRL.    
\end{itemize}
 
\DeclareGraphicsExtensions{.eps} %.eps
%\begin{figure}[Htb] 
\begin{figure}[H] 
\centering
\includegraphics[scale = 0.35]{figure1-SGP-algorithm} 
\caption{The scope of SGP algrithm with relaship to the related 
algorithms [3,4,5,6,10]. }
\label{figure1-SGP-algorithm} 
\end{figure} 

\section{Background on DRL Architecture}
In DRL archictecture as illustrated in Figure 2, depending
on the applications,  
we can use one of the Deep Neural Networks (DNN) models, 
e.g., SPG or DPG in the DRL system design. In our case, 
SPG is employed to deal with the stochastic nature of 
a 6 Degree-of-Freedom (DoF) robot control. 

\DeclareGraphicsExtensions{.eps} %.eps
%\begin{figure}[Htb] 
\begin{figure}[H] 
\centering
\includegraphics[scale = 0.35]{figure1-DRL-architecture} 
\caption{Use SPG in the DRL system design to deal with 
the stochastic nature of 
a 6 Degree-of-Freedom (DoF) robot control. }
\label{figure1-DRL-architecture} 
\end{figure} 

Note in the DRL architecture, there are 2 feedbacks, one feedback
is for the feedback of the agent state, or, system state 
denoted as ${S}$. This feedback is common in control systems. 
For example in PID (Proportional, Integral, Derivative) controller
design, we have feedback of those error functions,
or in modern control we have 
state feedback. However, the second feedback is quite 
unique, which is reward feedback. The reward feedback 
forms the core of reinforcement learning where the 
maximization of long term reward is the objective of 
deep learning process. 

In the context of robotic operations, 
a robot makes 
sequential movement, e.g., control actions in a stochastic 
environment. A stochastic environment makes 
robot action not necessarily bring to the desired 
position. The action only makes the robot move to 
a desired position with certain likelihood, which is 
described as probability distribution
$Prob(a_{t+1}|s_t,a_t)$. 

DRL mathematical formulation consists of 

1. a set S of states, denoted as 
$S = \{s_1, s_2, ..., s_N \}$, 
and 

2. a set A of actions, denoted as 
$A = \{a_1, a_2, ..., a_M \}$ 
as well as  

3. a transition function
$T$ and 

4. a reward function $R(s_t, a_t)$. 

We define   
a tuple $< S, A, T, R >$ for the above defined sets and 
functions. This tuple describes the relationship of 
the agents and its
interaction with an environment via state feedback
and reward functions. 

\subsection{Markov Decision Process}

Robot motion by the mathematical nature is a 
sequential decision process modeled as 
a Markov Decision Process (MDP) which consists of 
the tuple $< S, A, T, R >$, and has the following 
"short-memory" characteristics for its states $S$: 

\begin{equation}
Prob(s_{t+1}|s_t) = Prob(s_{t+1}|s_t, s_{t-1}, ..., s_2, s_1) . 
\end{equation}

Now, let's take a look at the reward function
$R(s_t, a_t)$ which define an one step reward 
at current state $s_t$ per the action $a_t$. 
Since the environment is stochastic, the agnet has its
next state $s_t$ stochastic as well, because the 
action $a_t$ may not lead the agent to the 
desired state $s_t$ as it is intended for. 
Therefore, the reward at this step
$r(s_t, a_t)$ is stochastic as well. 
Define a policy function 

\begin{equation}
{\pi}(a|s) = Prob(a|s, \theta) ,
\label{eq:policy-function-probability} 
\end{equation}
where $\theta$ is a set of agent (DNN) parameters. The 
policy $\pi$ is defined as a conditional probability
of an action $a$ given a state $s$ with the agent's 
DNN parameter $\theta$.

Choose Gaussian distribution for the policy function 
$\pi$. So to describe the entire process (e.g., an episode), 
for all rewards, we will have to take an average
of the rewards. This is the statistical expecation 
of the rewards as follows: 
$E_{s \sim P_{\pi} , a \sim {\pi}_{\Theta}} [ r(s, a) ]$. 

Denote this expectation as an objective function
$J({\theta})$, 
\begin{equation}
J({\theta}) = E_{s \sim P_{\pi} , a \sim {\pi}_{\Theta}} [ r(s, a) ]
\end{equation}
e.g.,  
\begin{equation}
J({\theta}) = 
\displaystyle
\sum_{s \sim S} Prob(s) \sum_{a \sim A} {\pi}_{\theta}(s,a) R (s, a) . 
\label{eq:objective-function-J} 
\end{equation}

When in any state $s \in S$ , an action $a \in A$ 
will lead to a new state with a transition 
probability $P_T ( s, a, s\prime )$ , and 
a reward R( s, a ) function.
In case of discrete actions $a_t$, we can 
have $a_t$ as left, or right, up or down in the 
case of driving a simulated car in a computer 
game. 

In the case of 6 degree-of-freedom 
(DoF) robot arm movement, we will have to operate with 
continued actions $a_t$. We use Stochastic 
Policy Gradient (SPG) technique. 
Given in this figure is 
a FD100 robot and it is operated in a 
stochastic environment and whose actions 
are continuous.  

\DeclareGraphicsExtensions{.eps} %.eps
%\begin{figure}[Htb] 
\begin{figure}[H] 
\centering
\includegraphics[scale = 0.35]{figure1} %name of the eps figure file 
\caption{An experimental platform for 
testing of the proposed ARP reward polic. An end effector 
(a set of three in this case) from CTI's FD100 robot, whose
control action is continuous.}
\label{fig1} 
\end{figure} 

\subsection{Stochastic Policy}
The objective for DRL is to maximize long term rewards, 
e.g., to maximize $J({\theta})$.  
Substitute equation ~\ref{eq:policy-function-probability},  
into equation 
~\ref{eq:objective-function-J}, so    
\begin{equation}
J({\theta}) = 
\displaystyle
\sum_{s \sim S} Prob(s) \sum_{a \sim A}Prob(a|s, \theta) R (s, a) . 
\label{eq:objective-function-J-probability} 
\end{equation}

We define policy by using Gaussian probability distribution
in equation ~\ref{eq:policy-function-probability}
as a stochastic policy. 

\subsection{Stochastic Policy Gradient}

Now, a gradient of $J({\theta})$, 

\begin{equation}
\bigtriangledown J({\theta})
= 
\displaystyle
\sum_{s \sim S} Prob(s) \sum_{a \sim A}
\bigtriangledown Prob(a|s, \theta) R (s, a) .
\label{eq:policy-gradient}
\end{equation}
Note: 
\begin{equation}
dx = x d(\log x) , 
\end{equation}
so is 
\begin{equation}
\bigtriangledown Prob(a|s, \theta) =
Prob(a|s, \theta) \bigtriangledown log (Prob(a|s, \theta)). 
\end{equation}
Therefore, we have 
\begin{equation}
\bigtriangledown J({\theta})
= 
\displaystyle
\sum_{s \sim S} Prob(s) \sum_{a \sim A}
Prob(a|s, \theta) \bigtriangledown log (Prob(a|s, \theta)) R (s, a) .
\label{eq:policy-gradient-log}
\end{equation}
which forms the foundation in this work. 

\section{Analysis of Reward Policy}
A trajectory of a robot motion defines 
a sequence of state-action pairs in time sequence 
$t_1, t_2, ..., t_N$, denoted as 
${\tau}_N = (s_1,a_1,s_2,a_2,...,s_N,a_N)$, defines 
the robot end effector's motion history in 3D space, 
whicd defines performance and is tied to a 
reward function.  
  
For the case of continuous actions in our study, 
the design of reward function $R(s_t, a_t)$ is defined 
based on the general guidelines
from Unity AI Robot github 
site [8,13]: 
 
\begin{itemize}
\item[1] Define the starting end effector's position as
$distance(0)$, or in short notation $d(0)$, 

\item[2] Define the target position as "NearestComponent"
as in the code implementation. The NearestComponent is 
a constant throughout each training episod. 

\item[3] Define distance $d(t)$ as the distance from 
the position of end effector to  
the NearestComponent, e.g., 
the target position [13]). 
$d(t) = || P(t) - P_{tgt} ||_2 $, 
where $P(t)$ is the position of 
the end effector at time $t$, and 
$P_{tgt}$ is the target position, which appears
in the Unity program as a position of 
the "NearestComponent". The target is fixed 
for each experiment. 
The distance $d(t)$ will be changing for each time 
index $t$ during the training. Note we use time
index $t$ here in the mathematical formulation, and 
index $i$ or $k$ in the program.
 
\item[4] Define $\Delta d(t)$ from the base line 
code derived from the 
RobotControllerAgent.c program [13]. 

\begin{equation}
\Delta d(t) = d(t) - PrevBest(t) . 
\end{equation}

\item[5] Denote PrevBest(t) as $P_b(t)$, so
\begin{equation}
P_b(t) = 
\begin{cases}
d(t) , & \text{if } d(t) < P_b(t) \\
P_b(t-1), & \text{o/w} .
\end{cases}
\label{eq:PrevBest(t)}
\end{equation}
So, we have 
\begin{equation}
P_b(t) = \min (d(t), P_b(t-1)) . 
\end{equation}
\end{itemize}

Now, let's take a look at the base line algorithm for the
reward function. 

Description of the base line algorithm 
\begin{itemize}
\item[1] For the extrem case I: 
when the arm hits the ground, a Hefty Penalty (-1) is 
given, e.g., 

\begin{equation}
R(s_t,a_t) = -1
\end{equation}
where $s_t =$ (ground state),  
then the training episode is terminated. 
Note each pisode is defined as one full cycle of training.

\item[2] For the extrem case II:
when the arm reaches the target, a Hefty Reward (1) 
is given,  
\begin{equation}
R(s_t,a_t) = 1
\end{equation}
where $s_{tgt} =$ (target state), then the training episode is 
terminated. 
\footnote{Or the training episod is ended
if a catastrophic event, e.g., robot 
end effector hits the ground
occurs.} 

The target state can be detected when the end effector 
$P(t)$ 
reaches the object(target) $P_{tgt}$, e.g., 
\begin{equation}
|| P(t) - P_{tgt} || \leq \epsilon
\end{equation}

\item[3] For the general cases, the base line
reward function is defined with an independent 
variable of $\Delta d(t)$ which forms the horizontal axis, 
the positive value of $\Delta d(t)$ reflects the increases 
of the distance of the robot end effector to its target 
position in a single time step, therefore it is not 
desirable behavior, or the penalty (negative reward). 

\begin{equation}
R(s_t,a_t) =
\begin{cases}
    - \Delta d (t), & \text{if $\Delta d (t)\geq 0$},\\
    Bd - Pb - \Delta d(t), & \text{$\Delta d (t)<0 $}.
\end{cases}
\label{eq:baseline_reward1}
\end{equation}
where 

\begin{equation}
Bd = BeginDistance(0) ,
\end{equation}
which is a distance set at 
time $t=0$ at the beginning of each episod and 
it will be a constant during the entire episod, and 
\begin{equation}
Pb (t) = PrevBest(t). 
\end{equation}
\end{itemize} 
 
Suppose it takes $N$ steps for the 
robot end effector to move from 
its current position $P(t)$ to $P_{tgt}$, e.g., 
$(s_t, a_t)$ for $t=1,2, ..., N$, if 
$P(t) = P_{tgt}$, the program will end the training episode
if

\begin{equation}
|| P(t) - P_{tgt} || \leq \epsilon
\end{equation} 

 
To analyze the current reward function from the 
Unity AI code [13], we introduce time index for 
each of the following 5 parameters to keep track 
the short past, current and possible near future
values:   
 
\begin{itemize}
\item[1] $beginDistance[i]$ of the end effector; 	
\item[2] $prevBest[i]$ of the end effector; 
\item[3] $distance[i]$ ; 
\item[4] $Delta_distance[i]$ ;	
\item[5] $reward[i]$ . 
\end{itemize}

\section{Accelerated Reward Policy (ARP)}
The reward function from Unity implementation based on the 
published work on [8, 13] 
adopt the distance based reward function, which is a linear 
reward function with the bigger reward for larger distance 
of the robot end effector traveled towards the target. 
We define this reward function as Type I reward \ref{eq:baseline_reward1}. 

To encourage positive behavior during the DRL training, 
for the bigger distance traveled towards the target by
the end effector, we design an
accelerated reward policy (ARP) by 
a new reward function as follows and illustrated in the 
following figure as well:    

\begin{equation}
R(s_t,a_t) =
\begin{cases}
    - K_1 \Delta d (t), & \text{if $\Delta d (t)<0$},\\
    K_2 (Bd - Pb - \Delta d(t)), & \text{$\Delta d (t)\geq 0$}.
\end{cases}
\label{eq:baseline_reward1}
\end{equation}

As you can see the new reward function
with $K_2$ gain coefficient on the nagative 
axis of $\Delta d(t)$ together with the base line 
existing slop of the reward function, denoted as $K_1$,
form a non-linear reward function (pice-wise linear).   

\begin{figure}[H] 
\centering
\includegraphics[scale = 0.35]{figure1-ARP} 
\caption{The new reward function
with $K_2$ gain coefficient on the nagative 
axis of $\Delta d(t)$ together with the base line 
existing reward function.}
\label{figure1-ARP} 
\end{figure} 

\section{Experiment Design}
\subsection{ARP Experiments}
To test the proposed ARP function, we conducted 
20 training experiments separated into 2 groups
with each of 10. Group 1 is for the base line algorithm 
and Group 2 is for the ARP function. 

Each experiment is conducted on Unity as shown below. 
The program is coded in CS (C\#), the DRL engine is 
in Python, and each experiment
starts with the same fixed starting 
position and same fixed target position for the 
robot end effector. We choose the experiments
with training upto 5,5000 steps. 
The code of our proposed algorithm was placed in the 
github [14,15,16]. 

\begin{figure}[H] 
\centering
\includegraphics[scale = 0.30]{figure1-unity} 
\caption{The experiments are conducted on Unity  
and the algorithm is coded in CS (C\#).
We choose each experiment
with training upto 5,5000 steps.}
\label{figure1-unity} 
\end{figure} 
 
 
\subsection{Re-write The Base Line Code}
We define the following parameters explicitly. 

\begin{itemize}
\item[1] {\it EndEff[i]} which defines the robot end effector's 
position in $x_w-y_w-z_w$ space. Clearly define {\it EndEff[i]} 
in $x_w-y_w-z_w$ space, 
e.g., $P_{end} = EndEff[i] = (x_{ef}(i),y_{ef}(i),z_{ef}(i))$, 

\item[2] Seperate relative movement from absolute 
movement. See figure below for illustration. The relative
movement is the movement from {\it EndEff[i-1]} to 
{\it EndEff[i]}, while the absolute movement is the movement 
defined with reference in $x_w-y_w-z_w$ space, simply stated, 
it is {\it EndEff[i]}. 
\end{itemize}  


\section{Performance Evaluations}
Compute the averaged reward from the base-line algorithm and 
the proposed ARP algorithm. Each of these 2 accumulated rewards
plotted with steps as an independent variable. We illustrated 
one of the accumulated reward function in the following figure. 
Note we define the steps which give the negative reward as 
region $\Omega_N$ while the steps give the positive reward
as region $\Omega_P$, the cross-over point is the point to 
separate these two regions.   
\begin{itemize}
\item[1] For a better performance, we are expecting to have 
a smaller accumulated negative reward, and  
\item[2] a bigger accumulated positive reward, e.g., the 
bigger area in the region of $\Omega_P$. 
\end{itemize}
\subsection{Index for Accumulated Reward}
Use the averaged outpout from both base-line algorithm and 
the proposed algorithm to calculate the accumulated 
reward as follows: 
\begin{equation}
%\int_{a}^{b} r_1(t;s_t, a_t) \,dt 
\int_{\Omega_{N}} r_1(t;s_t, a_t) \,dt 
+
\int_{\Omega_{P}} r_1(t;s_t, a_t) \,dt  
\label{eq:int-accumulated-reward} 
\end{equation}
Computationally, we have 
\begin{equation} 
I_N  
=
\sum_{t=1}^{t_k} r_1(t;s_t, a_t) , \text{      } I_P 
=
\sum_{t=t_{k+1}}^{t_N} r_1(t;s_t, a_t).  
\label{eq:indexP-discrete-accumulated-reward} 
\end{equation}

So we have $I_{N,K2}$ as the index 
covering the negative accumulated reward for the 
proposed ARP algorithm and $I_{N,B}$ as the index
of the accumulated reward for the base-line algorithm. 
Therefore, the indices are defined to compare the 
performance of two different reward functions as 
\begin{equation} 
\eta_N 
= \frac{I_{N,K2}}{I_{N,B}} , \text{      } \eta_P
= \frac{I_{P,K2}}{I_{P,B}} . 
\label{eq:EtaN-discrete-accumulated-reward} 
\end{equation}

\subsection{Performance Comparison}  
To compare the performance,
we have the following guideline.

If $\eta_N < 1$, then the accumulated negative reward
of the proposed ARP algorithm performs better in 
the sense of 
generating lesser negative penalty. And if 
If $\eta_P > 1$
then the accumulated positive reward
of the proposed ARP algorithm performs better in 
generating more positive reward.
 
\section{Analysis and Conclusion} 
The experimental result as shown in the following 
figure. The Illustration of the base line reward function
result (above) vs. the accelerated reward policy (ARP)
result. Note the very visible accumulative reward function 
increases for the ARP shown as a black color curve
in the lower half of this figure.

\begin{figure}[H] 
\centering
\includegraphics[scale = 0.35]{figure1-result.eps} 
\caption{Illustration of base line reward function
result vs. the accelerated reward policy (ARP)
result. Note very visible accumulative reward function 
increases for the ARP.}
\label{figure1-result} 
\end{figure}

\begin{figure}[H] 
\centering
\includegraphics[scale = 0.25]{figure1-result2.eps} 
\caption{Comparison of base line reward function
result (gray, lower) vs. the accelerated reward policy (ARP), 
(red, upper), 
with clear incease in  
accumulative reward function for the ARP.}
\label{figure1-result2} 
\end{figure}
There are 11 plots in each group, which are 10
different experiments and 1 calculated average as a 
mean value for the comparison to the other group.  
Note from step 0 to around 11,000 steps for Group 1
in Fig \ref{figure1-result}, 
the agent (robot) has negative reward. Then it
crosses over to the positive side of the Reward. 
The agent starts to receive a positive reward more after 11,000 
steps, the cross-over point. This trend of elevation continues 
gradually until around 55,000 steps. 
We neglect the steps after 55,000 steps as there are some missing 
training points due to 
there are a few curves with exactly 60,000 training steps on the 
graph.
Now, for the ARP algorithm evaluation,
we did same calculation of average curve from all 10 curves. 
For the 10 experiments of ARP 
with $K_1 =1.0$ and $K_2 =1.1$, we did not 
observe much visible difference from the experiments graph. 
Once we adjust the gain coefficients with $K_1 =1.0$ and 
$K_2 =2.0$, large visible change 
has occurved as shown in the lower half of this figure. 

From these 20 experiments with each of over 55K steps, 
we compute the reward and have the results listed in the 
following table. 
\begin{table}[H] % place table here 
\renewcommand{\arraystretch}{1.3} 
\caption{Experiment Results Comparison}
\label{Table1 Experiment Results Comparison}
\centering 
\begin{tabular}{|c||c||c|}
\hline
\hline 
Category     & Base line     & Accelerated Reward Policy    \\
\hline
Cross over   & 13,635 Steps & 6,529 Steps  \\
\hline
$I_N$  & $-5.21^{6}$ & $-1.1^{6}$ \\
\hline 
$I_P$  & $1.53^{8}$ & $3.97^{8}$ \\
\hline 
\hline 
\end{tabular}
\end{table}
Compute the performance based on equation 
\ref{eq:EtaN-discrete-accumulated-reward}, 
\begin{equation}
\eta_N = 0.212 \text{  and    } \eta_P = 2.595 . 
\end{equation}
These experiments results demonstrated 
\begin{itemize}
\item[1] The proposed ARP reward policy/function 
has converged faster, e.g., from 13,635 steps of the 
base line algorithm to 6,529 steps of the new 
algorithm on average. 
\item[2] The proposed ARP reward policy/function 
reduces the negative behavior including the catastrophic 
event (e.g., the end effector hitting the ground)
by significant reduction, from $-5.21^{6}$ of the base line 
to $-1.1^{6}$, about 21\% of less negative behavior. 
So the ARP has reduced the negative behavior of the 
robot agent by as much as 79\% from the experimental 
data results. 
\item[3] The ARP 
accelerates the robot learning/gainning desired 
movement from $1.53^{8}$ of the base line 
to $3.97^{8}$, about 2.59 times bigger in 
accumulative reward. 
Hence the ARP has accelerated the 
robot agent learning in a very visible way. 
\item[4] Since 20 experiments were conducted and each with 
steps upto 55K, it is needed to take this preliminary
results here to move to large experiment population with
each exmperiments going up to as many as 500K steps or 
even higher if needed for further verification. 
The code of our ARP algorithm was placed in the 
github [14,15,16]. 
\end{itemize}

\section*{Acknowledgment}
I would like to thank CTI One engineers and 
intern team for coding and carrying out robot parts 
design (by Allen Lee). 

\begin{thebibliography}{1}
\bibitem{1} 
Andrea Franceschetti, Elisa Tosello, Nicola Castaman, and Stefano
Ghidoni, 
"Robotic Arm Control and Task Training
through Deep Reinforcement Learning", 
https://arxiv.org/pdf/2005.02632.pdf, May 2020. 
 

\bibitem{2}  
https://github.com/Unity-Technologies/ml-agents 2020. 

\bibitem{3}
Tuomas Haarnoja, Sehoon Ha, Aurick Zhou, Jie Tan, Kristian Hartikainen, 
Vikash Kumar, Pieter Abbeel, Henry Zhu, George Tucker, Abhishek Gupta, Sergey 
Levine
UC Berkeley, and Google Brain, "Soft Actor-Critic Algorithms and Applications", 
https://arxiv.org/pdf/1812.05905.pdf, Jan. 2019. 

\bibitem{4}
Tuomas Haarnoja, Aurick Zhou, Pieter Abbeel, Sergey Levine, 
Berkeley Artificial Intelligence Research, University of California, Berkeley, 
USA.,
"Soft Actor-Critic: Off-Policy Maximum Entropy Deep Reinforcement
Learning with a Stochastic Actor", 
haarnoja@berkeley.edu, 
http://proceedings.mlr.press/v80/haarnoja18b/haarnoja18b.pdf, 2018. 

\bibitem{5} Petros Christodoulou, Imperial College London, 
"SOFT ACTOR-CRITIC FOR DISCRETE ACTION SETTINGS", 
petros.christodoulou18@imperial.ac.uk, https://arxiv.org/pdf/1910.07207.pdf, 
Oct. 2019.

\bibitem{6}
Tuomas Haarnoja, Aurick Zhou, Pieter Abbeel, Sergey Levine, 
Berkeley Artificial Intelligence Research, University of California, Berkeley, USA.,
"Soft Actor-Critic:
Off-Policy Maximum Entropy Deep Reinforcement
Learning with a Stochastic Actor", 
haarnoja@berkeley.edu, 
http://proceedings.mlr.press/v80/haarnoja18b, 
Aug. 2018. 

\bibitem{7}
Russ Salakhutdinov, Machine Learning Department, CMU,  
"10703 Deep Reinforcement Learning and Control", 
www.cs.cmu.edu, 2018. 

\bibitem{8}
The Unity Machine Learning Agents Toolkit, software,  
https://github.com/Unity-Technologies/ml-agents. 

\bibitem{9}
The experiments were conducted on Linux x86 machine with GPU
acceleration, and Mono environment for CS script is established
for programming and testing CS script which is utilized to 
modify the existing ML algorithm on the Unity ML platform. 

Mono, CS-Script engine file on Linux, 
www.cs-script.net.  

Q-Learning and difficulties with continuous action space
Value-Based Methods like DQN have achieved remarkable breakthroughs in the domain of Reinforcement Learning. However, their success is bound to problems with discrete action spaces, like Atari games.

\bibitem{10}
Takuma Seno, 
NAF: Normalized Advantage Function — DQN for Continuous Control Tasks, 
google search: DQN\_NAF+deep+reinforcement+learning, 
Oct. 2017.   

\bibitem{11}
Robot modeling on unity,  
https://blogs.unity3d.com/2020/11/19/robotics-simulation-in-unity-is-as-easy-as-1-2-3 

\bibitem{12}
3D model of Robot arm is from this link. The Look and feel of the model and hierarchy of components altered to fit for ML training purpose.

Joel, 3D Robot Modeling, esp. for the bottle pick and place model, 
/sketchfab.com/3d-models, 
https://sketchfab.com/3d-models/robot-arm-22d9367c8d2f4457b3a4e74193e86ac9. 

\bibitem{13}
in our training and simulation, we have employed the software for the
training of 6 axis robot arm inverse kinematics using Unity ML Agents

R. Kandas, RobotArmMLAgentUnity, https://github.com/rkandas. 

\bibitem{14}
Harry Li, Chee Vang, Shifa S., Base-line algorithm, for the 
Deep Reinforcement Learning based on the baseline reward function 
https://github.com/hualili/robotics...fd100/105g-1RobotControllerAgent-1Base.cs

\bibitem{15}
Harry Li, Chee Vang, Shifa S., Accelerated reward policy (ARP) algorithm, 
for the Deep Reinforcement Learning based on the ARP reward function
with K=2,   https://github.com/hualili/robotics- ...
fd100/105g-2RobotControllerAgent-2timeInd.cs 

\bibitem{16}
Harry Li, Chee Vang, Shifa S., Accelerated reward policy (ARP) algorithm, 
for the Deep Reinforcement Learning based on the ARP reward function
with bigger K,    
https://github.com/hualili/robotics- ...
fd100/105g-3RobotControllerAgent-3K2.cs

\bibitem{17}
Unity ML, the training configuration file and the definition of 
the NN architecture, 
https://github.com/Unity-Technologies/ml-agents/blob/main/docs/Training-Configuration-File.md 


\bibitem{18}
Mohammad Taghi Saffar, Mohammad Babaeizadeh, Danijar Hafner, Harini Kannan, 
Sergey Levine, Chelsea Finn, Dumitru Erhan, Google Brain, 
msaffar@google.com, mbz@google.com, danijar@google.com, 
hkannan@google.com, slevine@google.com, chelseaf@google.com, dumitru@google.com, 
"MODELS , PIXELS , AND REWARDS:
EVALUATING DESIGN TRADE-OFFS IN
VISUAL MODEL BASED REINFORCEMENT LEARNING", 
https://arxiv.org/abs/2012.04603,
Dec. 2020. 

\bibitem{19} 
Timothy P. Lillicrap, Jonathan J. Hunt, Alexander Pritzel, Nicolas Heess, 
Tom Erez, Yuval Tassa, David Silver, Daan Wierstra, 
"Continuous control with deep reinforcement learning", 
https://arxiv.org/abs/1509.02971, Sept. 2015. 

\bibitem{20}
Volodymyr Mnih, Koray Kavukcuoglu, David Silver, Alex Graves,
Ioannis Antonoglou, Daan Wierstra, Martin Riedmiller, Google 
DeepMind Technologies, 
"Playing Atari with Deep Reinforcement Learning"
{vlad,koray,david,alex.graves,ioannis,daan,martin.riedmiller}@deepmind.com, 
https://www.cs.toronto.edu/~vmnih/docs/dqn.pdf, 2015. 

\bibitem{21}
Volodymyr Mnih1, Koray Kavukcuoglu, David Silver, Andrei A. Rusu, 
Joel Veness, Marc G. Bellemare, Alex Graves,
Martin Riedmiller, Andreas K. Fidjeland, Georg Ostrovski, Stig Petersen, 
Charles Beattie, Amir Sadik, Ioannis Antonoglou,
Helen King, Dharshan Kumaran, Daan Wierstra, Shane Legg, and Demis Hassabis
"Human-level control through deep reinforcement
learning", LETTER doi:10.1038/nature14236, Nature, pp. 529., Feb. 2015,   

\end{thebibliography}
\end{document}


